\section{Introduction}\label{introduction}

\subsection{Motivation}\label{motivation}
The work done in this report is done in cooperation with Sensomar SEALAB.
Sensomar SEALAB is a company developing camera systems for underwater use. Most of this is planned to be used in the fishing industry, especially aimed towards salmon breeding industry. The company's current goal is to be able to automatically measure the size and weight of the salmon underwater. For this goal to be reached, SEALAB uses some of the best cameras available, that has very high resolution and a very accurate depth measurement. But, due to all particles in the water, other fish in the background and light reflections, it is still not possible to use the data directly to find accurate measurements of the fish. They therefore need to both detect the fish, to be able to take a good photo, and they will need quite a bit of image processing on that image to remove all irrelevant data before they can find the measurements of the fish. 

With the use of image processing for noise and particle removal, it is reasonable to believe that the size measurements of the fish could be improved.

This report will show relevant background study with explanations to the different techniques that have been tried out. It will show implementation and results from different approaches. All image processing done throughout this project is done through the use of C++ and OpenCV library. OpenCV (Open Source Computer Vision) is a library of programming functions mainly aimed at real-time computer vision. \footnote{https://en.wikipedia.org/wiki/OpenCV}

\subsection{Image processing simple intro??}

\subsection{Camera used??}
