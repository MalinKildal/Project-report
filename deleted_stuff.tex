
\subsection{Introduction to Digital Image Processing}
{\color{red} REMOVE section}
 
Digital images are represented as a two-dimensional function, f(x,y), where x and y are plane (spatial) coordinates each representing one pixel. Each pixel has a finite value that represents the intensity, colors, depth, gray levels etc. To represent a color, OpenCV use the BGR color space. That means that each pixel contains an array with 3 values, [Blue, Green, Red], where the values are normally an integer between 0 and 255. The BGR color space is shown in figure \ref{fig:bgr_color_space}.

\begin{figure}[h]
    \centering
    \includegraphics[width=.9\linewidth]{images/introduction/RGB_color_solid_cube}
    \caption{BGR color space}
    \label{fig:bgr_color_space}
\end{figure}

Digital image processing is the use of computer algorithms to perform some processing on a digital image. Through the use of image processing it is possible to achieve simple things as changing the color space, to more advanced things as object recognition and scene understanding. Digital image processing is used today for important tasks as medical visualisation and industrial inspection, but also just as filters on Instagram to make someone look tanner. 

Digital image processing has many areas of use, listed below are some of the most important ones:
\begin{itemize}
\item Visualization - observe objects that are not visible.
\item Classification - decide what kind of object is in the image.
\item Image sharpening or restoration - enhance the image quality.
\item Pattern recognition - used in machine learning to recognize patterns and regularities in the image data.
\item Projection - image of a three-dimensional object is projected into a planar surface. Used in technical drawing.
\item Linear filtering - filter out data from images.
\end{itemize}

\cite{book:digital_image_processing, book:machine_vision}

{\color{red}Add image examples???}






{\color{red}
Innledning: Create a research space (CARS)
• Presenter forskningsområdet («establish territory»)
    • Presenter tema og kunnskapsstatus i feltet
• Vis behov/nødvendigheten for din forskning («establish niche»)
    • Bygg på tidligere forskning
    • Vis at kunnskap er manglende/fraværende
    • Vis spørsmål/konflikter ved tidligere forskning
• Presenter din forskning («occupy niche»)
    • Hva vil du oppnå?
    • Uttrykt gjennom hypotese, formål eller forskningsspørsmål
    • På hvilken måte kan prosjektet ditt bidra til ny kunnskap?


Handler altså om: HVA og HVORFOR
Beskriv konteksten/feltet/bakgrunn – og motiver egen forskning (Hva og Hvorfor)

• Hva
    • Hva handler prosjektet ditt om (tema)?
    • Hva gjør du i prosjektet ditt?
    • Hva ønsker du å oppnå/finne ut?
• Hvorfor (motivasjon)
    • Hvorfor er det viktig å undersøke/skrive om det?
    • Har andre undersøkt problemet? Hva har de funnet?
    • Hva skal prosjektet ditt bidra til – hvilket «kunnskapshull» skal det fylles?


Undersøke innledningen
• Hvordan er innledningen delt inn? Hvilke underkapitler?
• Hva gjør skriveren i første avsnitt?
    • «Beveger» teksten seg «fra det generelle til det spesifikke» – og på hvilken måte?
• Ser du kildehenvisninger – hva brukes de til?


}




\subsection{Different Levels of Particle Noise}

To make the algorithm suitable for different conditions of water, noise was added to the depthmap.
The original plan was for the water to get blurry by it self since the dead salmons rotting process is very quick. The water, at the end of picture taking, looked very blurry and filled with particles, but it did not show much differences in the depthmap image. It's not known if it is mostly the water mediums properties that causes noise in the depthmap, and that particles are not enlarging this disorder significantly, or if enough particles within reason will disturb the depth measurements done by the Raytrix.

Since we could not get the naturally disturbances in the water to affect the depthmap directly, noise was added to the computed depthmap images.

The noise used is salt and pepper noise, which turns a random set of pixels into either black or white. For the noise to illustrate real particles in different size and shape, morphological closing was added to the salt and pepper noise before the mask was added to the depthmap image.


\subsubsection{Level 1}

Noise level 1 is the original images taken by the Raytrix camera, without extra noise added.
The result from these images can be seen in section \ref{section:depthmap}.


\subsubsection{Level 2}

Noise level 2 has some salt and pepper noise added to it. Figure \ref{fig:noise_level_2} shows the resulting depthmap and the result using the same approach as described in section \ref{section:depthmap}.

\begin{figure}[h]
    \centering
    \begin{subfigure}{\textwidth}
        \centering
        \includegraphics[width=.7\linewidth]{images/implementation/noise_level_2/1_original}
        \caption{Depthmap image with noise level 2} 
        \label{fig:noise_level_2_original}
    \end{subfigure}\hspace*{\fill}
    
    \medskip
    \begin{subfigure}{\textwidth}
        \centering
        \includegraphics[width=.7\linewidth]{images/implementation/noise_level_2/7_median_filter}
        \caption{Resulting depthmap} 
        \label{fig:noise_level_2_result}
    \end{subfigure}\hspace*{\fill}
    \caption{Depthmap image and result for noise level 2}
    \label{fig:noise_level_2}
\end{figure}


\subsubsection{Level 3}

Noise level 3 has more salt and pepper noise that noise level 2, and the noise is morphologically closed more times. The noisy image and the result after using the algorithm from section \ref{section:depthmap} can be seen in figure \ref{fig:noise_level_3}.

\begin{figure}[h]
    \centering
    \begin{subfigure}{1\textwidth}
        \centering
        \includegraphics[width=.7\linewidth]{images/implementation/noise_level_3/1_original}
        \caption{Depthmap image with noise level 3} 
        \label{fig:noise_level_3_original}
    \end{subfigure}\hspace*{\fill}
    
    \medskip
    \begin{subfigure}{1\textwidth}
        \centering
        \includegraphics[width=.7\linewidth]{images/implementation/noise_level_3/7_median_filter}
        \caption{Resulting depthmap} 
        \label{fig:noise_level_3_result}
    \end{subfigure}\hspace*{\fill}
    \caption{Depthmap image and result for noise level 3}
    \label{fig:noise_level_3}
\end{figure}
