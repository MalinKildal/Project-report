\begin{abstract}
\addcontentsline{toc}{section}{\bf Abstract}

This report discusses challenges faced by biomass estimation of underwater objects. This is an interesting topic, as a technological innovative solution could be important for the aquaculture industry, especially the fish breeding industry. 

The objective of this study is using the Raytrix plenoptic technology for attaining underwater 3D images of objects and improve the results of the generated color depthmap. This is important because there is no existing documentation on the Raytrix for underwater purposes, and the 3D data generated by the Raytrix underwater often lacks important data. This study's main focus is to remove particle noise form the Raytrix color depthmaps and fill in sports of missing color data with appropriate data. 

This report presents the Raytrix plenoptical technology along with some theory on different useful image processing techniques. A suggested implementation is provided, and results from this implementation are presented and discussed.

\end{abstract}


