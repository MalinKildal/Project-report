\section{Future Work} \label{future_work}

The work done in this report is just a small part towards the goal of a complete underwater volume measurement system using Raytrix plenoptic technology. 

Further development would require more image testing for a better base image material from the Raytrix. The most important test condition is probably light. Therefore, different light conditions should be tested both in a closed tank and in open sea. With good conditions better base images would be established and that would make much of the further work a lot easier. 
When dealing with fish as objects, especially salmon, it is hard to achieve good light as the fish reflects light coming directly towards it. It is also a problem that the fish's back along with its fins and head are very black, such that the Raytrix have problem detecting differences in the micro lenses, and thereby generates poor depth data.
This was tested during this project, but due to lack of resources and time all wanted testing was not obtained.

Another important step is merging the Raytrix calibration data into the color depthmap, or somehow establish an interaction between the color depthmap and the Raytrix depth data. This is something that must be achieved if volume measurement is to be realized.

The next important part is implementing an algorithm that actually measures the volume of the objects. This is most likely a step that requires much testing, adjustments and verification, starting with simple objects before testing on more advanced shapes. 

