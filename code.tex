\section{Code} \label{code}

\definecolor{codegreen}{rgb}{0,0.6,0}
\definecolor{codegray}{rgb}{0.5,0.5,0.5}
\definecolor{codepurple}{rgb}{0.58,0,0.82}
\definecolor{backcolour}{rgb}{0.95,0.95,0.92}

\lstdefinestyle{customc}{
    language=C++,
    backgroundcolor=\color{backcolour},   
    commentstyle=\color{codegreen},
    keywordstyle=\color{magenta},
    numberstyle=\tiny\color{codegray},
    stringstyle=\color{codepurple},
    basicstyle=\footnotesize,
    breakatwhitespace=false,         
    breaklines=true,                 
    captionpos=b,                    
    keepspaces=true,                 
    numbers=left,                    
    numbersep=5pt,                  
    showspaces=false,                
    showstringspaces=false,
    showtabs=false,                  
    tabsize=2
}

\lstset{escapechar=@,style=customc}



\subsection{main.cpp}
\lstinputlisting[style=customc]{Files/main.cpp}


\subsection{FindLargestContour.cpp}
The method used for finding the largest contour is a modified version of a method from GitHub \cite{website:largest_contour_code_github}.
\lstinputlisting[style=customc]{Files/FindLargestContour.cpp}


\subsection{Filters.cpp}
\lstinputlisting[style=customc]{Files/Filters.cpp}

