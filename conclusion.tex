\section{Conclusion}\label{conclusion}

The objective of this study was to enhance the depthmap of underwater objects computed by the plenoptic camera Raytrix R42, so it can further be used for a more accurate volume measurement of the object. 

In order to achieve this, an image processing algorithm containing color filtering, morphological operators, object detection and different filtering techniques was introduced. All processing was carried out in RxLive, C++ and OpenCV. MATLAB was used as a visualization tool for better understanding of the results and its improvements. The results showed a great improvement of the depthmap image, with most particles removed and all spots lacking data filled in with appropriate data. MATLAB plots also proves the preservation of depth data.

Different levels of noise were introduced as to illustrate how well the proposed method handles particle noise. The methods implementation was not changed during noise testing, and it showed that the method performed surprisingly well in harsh underwater conditions. 

The results attained from this report were promising as to the further development of volume measuring of underwater objects using the Raytrix plenoptic technology. It is reason to believe that the method developed from this study can be part of a complete future underwater system detecting and measuring biomass of fish in fish farms. 


