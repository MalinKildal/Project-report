
\title{Image Enhancement Techniques for Underwater Images}
\author{Malin Kildal}
\date{\today}

\documentclass[11pt]{article}

\usepackage{graphicx}
\usepackage{color}
\usepackage{caption}
\usepackage{subcaption}
\usepackage{url}
\usepackage{enumitem}
\usepackage{mathtools}


\makeatletter % Set distance from top of page to first float
\setlength{\@fptop}{5pt}
\makeatother


\begin{document}

\begin{titlepage}

\newcommand{\HRule}{\rule{\linewidth}{0.5mm}}
\center

\textsc{\LARGE NTNU}\\[1.0cm]
\textsc{\LARGE Sensomar SEALAB}\\[3.0cm]
%\textsc{\Large Major Heading}\\[0.5cm]
%\textsc{\large Minor Heading}\\[0.5cm]

\HRule \\[0.5cm]
{ \huge \bfseries Image Enhancement Techniques \\
for Underwater Images}\\[0.5cm]
\HRule \\[3.0cm]

\begin{minipage}{0.4\textwidth}
\begin{flushleft} \large
\emph{Author:}\\
Malin Kildal
\end{flushleft}
\end{minipage}
~
\begin{minipage}{0.4\textwidth}
\begin{flushright} \large
\emph{Supervisor:} \\
Annette Stahl
\end{flushright}
\end{minipage}\\[6cm]

{\large \today}\\[1cm]

\end{titlepage}



\begin{abstract}
This report discusses how how to improve underwater images of salmon. This is a relevant problem for the salmon breeding industry, as further image processing could be used to measure the size and weight of the fish. It could also be an opportunity to detect sick fish.

The main focus of this study is to remove irrelevant data from the image, while preserving the main object.
The image material will be both 2D and 3D, and the goal is for the size measurements to be more accurate after this image processing than before.

The report will show different image processing techniques that can be used to remove particles and noise. To measure the weight of the fish, both length, height and thickness measurements are needed. To achieve accurate measurements the result should be a clear image, where the depth measurements shows a smooth surface on the fish.

{\color{red}
Viktige elementer i et abstrakt:
\begin{itemize}
    \item Bakgrunnsinformasjon
    \item Formålet med studien
    \item Informasjon om metoden
    \item Viktigste resultatene
    \item Konkluderende utsagn
\end{itemize}}

\end{abstract}
\clearpage


\tableofcontents
\clearpage


\section{Introduction}\label{introduction}

\subsection{Motivation}\label{motivation}
The work done in this report is done in cooperation with Sensomar SEALAB.
Sensomar SEALAB is a company developing camera systems for underwater use. Most of this is planned to be used in the fishing industry, especially aimed towards salmon breeding industry. The company's current goal is to be able to automatically measure the size and weight of the salmon underwater. For this goal to be reached, SEALAB uses some of the best cameras available, that has very high resolution and a very accurate depth measurement. But, due to all particles in the water, other fish in the background and light reflections, it is still not possible to use the data directly to find accurate measurements of the fish. They therefore need to both detect the fish, to be able to take a good photo, and they will need quite a bit of image processing on that image to remove all irrelevant data before they can find the measurements of the fish. 

With the use of image processing for noise and particle removal, it is reasonable to believe that the size measurements of the fish could be improved.

This report will show relevant background study with explanations to the different techniques that have been tried out. It will show implementation and results from different approaches. All image processing done throughout this project is done through the use of C++ and OpenCV library. OpenCV (Open Source Computer Vision) is a library of programming functions mainly aimed at real-time computer vision. \footnote{https://en.wikipedia.org/wiki/OpenCV}

%%%%%%%%%%%%%%%%%%%%%%%%%%%%%%%%%%%%%%%%%%%%%%%%%%%%%%%%%%%%%%%%%%%%%%

\subsection{Image processing simple intro??}

{\color{red}Add image examples}


%%%%%%%%%%%%%%%%%%%%%%%%%%%%%%%%%%%%%%%%%%%%%%%%%%%%%%%%%%%%%%%%%%%%%%

\subsection{The Raytrix Camera}

\begin{figure}[h]
    \centering
    \includegraphics[width=.9\linewidth]{Images/raytrix_camera}
    \caption{Raytrix R42 Camera}
    \label{fig:raytrix_camera}
\end{figure}

The camera used throughout this project is the Raytrix R42 camera. The camera is developed by Raytrix, a German company which offers several 3D Light Field cameras intended for professional and industrial use. The R42 is their highest resolving light field camera to date. It is based on a 42 megaray sensor and offers an effective resolution up to 10 megapixels at 7 fps. \footnote{https://www.raytrix.de/produkte/#r42series}

Light Field cameras are a new type of 3D-cameras that capture a standard image together with the depth information of a scene. Metric 3D information can be captured with a single light field camera through a single lens in a single shot using just the available light. Raytrix has specialized on developing light field cameras for industrial applications. A patented micro lens array design allows for an optimal compromise between high effective resolution and large depth of field. Raytrix cameras are already in use in applications like volumetric velocimetry, plant phenotyping, automated optical inspection and microscopy, to name a few. \footnote{https://www.raytrix.de/}

This camera is therefore very useful for industrial purposes, as you can get both a clear image, and a depth map with a 3D model of the scene. It has though no documented use underwater, and since the physical properties of water cause degradation effects not present in air, the depth measurements get affected. We get both "holes" in the fish, and particles in front of the fish. If we are going to use this camera technology to measure volume of fish, we need to to improve the data received from the camera. Particles in front of the fish needs to be removed, and holes in the fish filled.

{\color{red}Add image examples of 3D in raytrix}

As seen from the 3D image produced in the Raytrix Software, the particles in front will affect the surface of the fish. The same goes for the holes in the fish. What is wanted is a smooth surface of the fish, so that measurements of length, height and thickness is most accurate. 
One problem working with the depthmap images is that they have their own .ray format. It is not yet possible to do direct image processing using OpenCV on this format, so we therefore convert the depthmap images to .png images before starting to work on the enhancement.

{\color{red}Add image examples of color depth map}
\begin{figure}[h]
    \centering
    \includegraphics[width=.9\linewidth]{Images/depthmap_raytrix}
    \caption{Example of depthmap image}
    \label{fig:raytrix_depthmap}
\end{figure}

The depthmap image looks like the one i figure \ref{fig:raytrix_depthmap}. The depth is represented in colors, where red is close to the camera, yellow and green is the object centered during calibration, blue is behind the object, and black has no depth. This means that the optimal depthmap image would be all black, with a yellow fish.


{\color{red}Explain more about how the Raytrix measures depth?? Explain the calibration and the RAW image??}
\clearpage

\section{Overview of Literature}\label{overview}
This section presents literature containing background information on image processing, both background studies and already existing solutions to similar problems. There has not been any suggestions from existing literature on how to solve this exact problem, but there are many different approaches that solves parts of the problem.
Below follows articles and books providing theory and solutions for removing noise from underwater images.

\subsection{Digital Image Processing (Gonzalez and Woods, 2008)}
This book has been the leading book in its field for a long period of time. It gives a simple introduction to image processing, explaining the fundamentals and giving examples. It does not give complete solutions or complex algorithms, but it helps getting familiar with the basics and to understand how the computer sees images relative to how humans see them.

The book explains simple procedures as a averaging filter and median filter, which can be used to blur or sharpen the image. 

It also explains more complex procedures as morphological operators with eroding and dilating and Fourier filtering. 

Eroding and dilating can be used to sharpen an image, or it can be used to remove particles from the image by reducing the size of objects until the unwanted objects disappear and then again expand the main object until it reaches its original size. 

Fourier filtering shows how to transform the image from the spatial domain into the frequency domain. In the frequency domain it is possible to remove some frequencies, and then transform the image back to the spatial domain. This can be used to remove light scattering from an image, which can be useful to remove in some underwater images.

\subsection{Machine Vision (Jain, Kasturi and Schunck, 1995)}

\subsection{Underwater Image Processing: State of the Art of \\
Restoration and Image Enhancement Methods \\
(Raimondo Schettini and Silvia Corchs, 2010)}
The paper by Raimondo Schettini and Silvia Corchs explains the problems with underwater image processing due to the light propagation in the water medium. The physical properties of water cause degradation effects not present in images taken in air. The water absorbs light and therefore limits the visibility distance, and it also causes scattering, which changes the direction of the light path. This influences the overall performance of underwater imaging systems. Forward scattering is the spreading of light from an object towards the camera, while backward scattering is the light reflected by the water towards the camera before it actually reaches an object. Backward scattering reduces the contrast of the image, and is seen as sunbeams on the image. Scattering comes from not only the water, but all particles in the water. 

\begin{figure}[h]
    \centering
    \includegraphics[width=0.7\textwidth]{Images/image_from_paper}
    \caption{Explanation of scattering from (Raimondo Schettini and Silvia Corchs, 2010, p. 3)}
    \label{fig:image_from_paper}
\end{figure}

The paper explains different sub-problems and what approaches that can be used to solve them. There are many example images with clear improvement, but the paper does not contain any examples on implementation. Some of the techniques mentioned in the paper are MTF, ACE, Bazeille et al, Iqbal et al and Petit et al. The paper also contains a table of all relevant techniques, with a brief description and usage area.

{\color{red}Try out some of these techniques???}


\subsection{OpenCV Online Tutorial \\
(http://docs.opencv.org/2.4/doc/tutorials/tutorials.html)}
To learn about the OpenCV library and its functionality this online tutorial was quite useful. It contains many good examples on different implementations and how to use the functions in the OpenCV library. All functionality is explained very well with goal, theory on the topic, code, explanation of the code and results. These tutorials made it very easy getting started with OpenCV.

\clearpage

\section{Aim of the Study}\label{aim of study}

This section explains why the depthmap enhancement is needed, and how it would help the aquaculture industry. There is also a short section explaining the challenges met by working with the Raytrix camera and the depthmap images.

\subsection{Scope}

The goal of this project is to help SEALAB OCEAN GROUP on their way of making a complete camera system ready for underwater use for the aquaculture industry. SEALABs complete system should be able to detect the fish and classify it, measure the volume of the fish and detect salmon lice.
This projects focus is to better the depthmap images taken by the Raytrix camera so that the volume measurement becomes more accurate. Under the proper light conditions and at some certain distance, this individual system should be able to measure the volume of an object underwater. The task is aimed towards fish, especially salmon, as this system would mostly benefit the salmon breeding industry.
The aquaculture fish breeding industry is currently selling their fish without any knowledge on how much they actually got. With a complete camera system in place in every fish farm, an estimate of the total volume of fish can be made, salmon lice can be discovered earlier, and the companies can save huge amounts of money. 

The aim of this study is for the volume estimation of salmon to become more accurate. For this to happen, particles and unnecessary data needs to be removed and be replaced by relevant data. The depthmap computed by the Raytrix camera normally looks like the one in figure \ref{fig:depthmap82}, and it is seen from this image that there are holes in the fish, parts of the fish's belly and head is missing, and there are particles in front of the fish. This gives large opportunities for improvement, but also some challenges that need to be worked out.

\begin{figure}[H]
    \centering
    \includegraphics[width=.7\linewidth]{images/aim_of_study/depthmap82}
    \caption{Depthmap image of salmon}
    \label{fig:depthmap82}
\end{figure}


\subsection{Challenges}

The volume measurement is depending on the depth information stored in the depthmap image. For this measurement to be accurate, the data in the depthmap image must be more complete. Holes in the fish must be filled in while the shape of the fish remains.

A MATLAB 3D plot shows how insufficient the actual depth information is (see figure \ref{fig:matlab3D}). The particles make for large disturbances and the holes in the fish make for even bigger ones. The parts of the salmon that make for the largest depthmap errors is the belly, the fins and its head. The belly is white and the color is too close the the background color in the test facility. The head and fins are very black, and it could be a problem to the Raytrix that it is too black, and therefore makes it hard to see differences in each micro lens.

\begin{figure}[H]
    \centering
    \includegraphics[width=.7\linewidth]{images/aim_of_study/original_3D_82}
    \caption{3D plot of depthmap}
    \label{fig:matlab3D}
\end{figure}

Even though there are many errors, it should be possible using the data from the depthmap image, enhance it and get a smooth surface of the fish, and thereby get a successful volume measurement. That is what the following sections in this report hope to show using the theory provided in section \ref{theory}.

\clearpage

\section{Methods and Implementation}\label{methods and implementation}

This section presents different processing techniques with the aim of solving the task at hand. SEALAB's current solutions is explained and discussed. The new implementation is explained step by step followed by images showing the process and the results.


\subsection{Previous implementation done by SEALAB}

{\color{red}
Write about what they have already tried out. 
they have tried interpolation filling out the fish. Show picture of the green fish. 
This did work, but the interpolation algorithm take forever to run and you can see the colors that represent the depth of the fish is not that even.

What is also now done is get the p-data directly from the Raytrix software, then make a depthmap from the p-data and use the p-data in the volume algorithm. What I want to do is get the depthmap directly from the Raytrix, do some work on it, extract the p-data based on color, then send the data to through the volume algorithm. 
}

%%%%%%%%%%%%%%%%%%%%%%%%%%%%%%%%%%%%%%%%%%%%%%%%%%%%%%%%%%%%%%%%%%%%%%


\subsection{Depthmap enhancement} \label{section:depthmap}

\subsubsection{Morphological Closing and Object Detection}

The approach used to better the depthmap image used morphological closing on the depthmap to even out the color and fill the holes. It started with color-filtering to remove particles too close or too far behind the main object. The algorithm also found the largest contour from the original image to use as a mask. The algorithm followed the steps below. The complete code can be found in section \ref{code}.
\begin{enumerate}
    \item \textbf{Filter out unwanted colors:}
    Color-filtering was done by looping through each pixel in the image and setting it to black if its blue-value was too high, green-value too low or red-value too low. This removed all particles in the image that was too close or to far behind. 
    \item \textbf{Remove small particles:}
    Removal of small particles was done by eroding twice followed by dilating twice. Further removal would remove to much of the fish.
    \item \textbf{Find largest contour:}
    Finding the largest contour followed these steps: \cite{website:largest_contour_code_explanation}
    \begin{enumerate}[label*=\arabic*.]
        \item Convert to grayscale and normalize
        \item Sobel edge detector
        \item Dilate
        \item Floodfill
        \item Erode
        \item Find largest contour
    \end{enumerate}
    The resulting images is shown in figure \ref{fig:find_largest_contour}.
    \item \textbf{Morphological closing on color-filtered image:}
    To fill out the holes in the fish in an even and fast way, morphological closing was used.
    \item \textbf{Mask the morphological closed image with the largest contour:}
    Because the morphological closing operations would expand the fish, the largest contour was used as a mask to gain the final result. 
\end{enumerate}

Figure \ref{fig:algorithm} shows the steps done by the algorithm.

The algorithm was tested on different depthmap images, and gave very good results on many of them. It was also tested to find the largest contour from the totalfocus image instead of the depthmap, but the depthmap returned a more accurate contour. Figure \ref{fig:algorithm_test} shows the results from 4 different depthmaps on the same fish.
The result is good compared to the original depthmap, but the depthmap computed from the Raytrix software could be improved. This is most likely done by different lightning, or by giving the background a different color.
Painting the test facility black was tried, but it only resulted in missing spots on the fishes back instead of its bog.


\begin{figure}[H]
    \begin{subfigure}{0.48\textwidth}
        \includegraphics[width=\linewidth]{images/implementation/4_1_grayscale}
        \caption{Converted to grayscale} 
        \label{fig:grayscale}
    \end{subfigure}\hspace*{\fill}
    \begin{subfigure}{0.48\textwidth}
        \includegraphics[width=\linewidth]{images/implementation/4_2_edge_detector}
        \caption{Edge detection} 
        \label{fig:edge_detection}
    \end{subfigure}
    
    \medskip
    \begin{subfigure}{0.48\textwidth}
        \includegraphics[width=\linewidth]{images/implementation/4_3_dilate}
        \caption{Dilate} 
        \label{fig:dilate_contour}
    \end{subfigure}\hspace*{\fill}
    \begin{subfigure}{0.48\textwidth}
        \includegraphics[width=\linewidth]{images/implementation/4_4_floodfill}
        \caption{Floodfill} 
        \label{fig:floodfill}
    \end{subfigure}
    
    \medskip
    \begin{subfigure}{0.48\textwidth}
        \includegraphics[width=\linewidth]{images/implementation/4_5_erode}
        \caption{Erode} 
        \label{fig:erode_contour}
    \end{subfigure}\hspace*{\fill}
    \begin{subfigure}{0.48\textwidth}
        \includegraphics[width=\linewidth]{images/implementation/4_largest_contour}
        \caption{Largest contour} 
        \label{fig:largest_contour_2}
    \end{subfigure}
    \caption{Finding the largest contour} 
    \label{fig:find_largest_contour}
\end{figure}


\begin{figure}[H]
    \begin{subfigure}{0.48\textwidth}
        \includegraphics[width=\linewidth]{images/implementation/1_original}
        \caption{Original depthmap image} 
        \label{fig:original_depthmap}
    \end{subfigure}\hspace*{\fill}
    \begin{subfigure}{0.48\textwidth}
        \includegraphics[width=\linewidth]{images/implementation/2_color_filtering}
        \caption{Color-filtered image} 
        \label{fig:color_filtering}
    \end{subfigure}
    
    \medskip
    \begin{subfigure}{0.48\textwidth}
        \includegraphics[width=\linewidth]{images/implementation/3_remove_particles}
        \caption{Removal of small particles} 
        \label{fig:remove_particles}
    \end{subfigure}\hspace*{\fill}
    \begin{subfigure}{0.48\textwidth}
        \includegraphics[width=\linewidth]{images/implementation/4_largest_contour}
        \caption{Largest contour} 
        \label{fig:largest_contour}
    \end{subfigure}
    
    \medskip
    \begin{subfigure}{0.48\textwidth}
        \includegraphics[width=\linewidth]{images/implementation/5_closing_on_color_filtered_image}
        \caption{Morphological closing} 
        \label{fig:morphological_closing}
    \end{subfigure}\hspace*{\fill}
    \begin{subfigure}{0.48\textwidth}
        \includegraphics[width=\linewidth]{images/implementation/6_masked_source}
        \caption{Masked image} 
        \label{fig:masked_source}
    \end{subfigure}
    \caption{Morphological closing and masking with largest contour} 
    \label{fig:algorithm}
\end{figure}


\begin{figure}[H]
    \begin{subfigure}{0.48\textwidth}
        \includegraphics[width=\linewidth]{images/implementation/algorithm_test/original_63}
        \caption{Original depthmap image} 
        \label{fig:original_depthmap_63}
    \end{subfigure}\hspace*{\fill}
    \begin{subfigure}{0.48\textwidth}
        \includegraphics[width=\linewidth]{images/implementation/algorithm_test/median_filter_63}
        \caption{Result} 
        \label{fig:result_63}
    \end{subfigure}
    
    \medskip
    \begin{subfigure}{0.48\textwidth}
        \includegraphics[width=\linewidth]{images/implementation/algorithm_test/original_73}
        \caption{Original depthmap image} 
        \label{fig:original_depthmap_73}
    \end{subfigure}\hspace*{\fill}
    \begin{subfigure}{0.48\textwidth}
        \includegraphics[width=\linewidth]{images/implementation/algorithm_test/median_filter_63}
        \caption{Result} 
        \label{fig:result_73}
    \end{subfigure}
    
    \medskip
    \begin{subfigure}{0.48\textwidth}
        \includegraphics[width=\linewidth]{images/implementation/algorithm_test/original_82}
        \caption{Original depthmap image} 
        \label{fig:original_depthmap_82}
    \end{subfigure}\hspace*{\fill}
    \begin{subfigure}{0.48\textwidth}
        \includegraphics[width=\linewidth]{images/implementation/algorithm_test/median_filter_63}
        \caption{Result} 
        \label{fig:result_82}
    \end{subfigure}
    
    \medskip
    \begin{subfigure}{0.48\textwidth}
        \includegraphics[width=\linewidth]{images/implementation/algorithm_test/original_87}
        \caption{Original depthmap image} 
        \label{fig:original_depthmap_87}
    \end{subfigure}\hspace*{\fill}
    \begin{subfigure}{0.48\textwidth}
        \includegraphics[width=\linewidth]{images/implementation/algorithm_test/median_filter_87}
        \caption{Result} 
        \label{fig:result_87}
    \end{subfigure}
    
    \caption{Algorithm test on different images} 
    \label{fig:algorithm_test}
\end{figure}

\newpage


%%%%%%%%%%%%%%%%%%%%%%%%%%%%%%%%%%%%%%%%%%%%%%%%%%%%%%%%%%%%%%%%%%%%%%


\subsection{Different Levels of Particle Noise}

To make the algorithm suitable for different conditions of water, noise was added to the depthmap.
The original plan was for the water to get blurry by it self since the dead salmons rotting process is very quick. The water, at the end of picture taking, looked very blurry and filled with particles, but it did not show much differences in the depthmap image. It's not known if it is mostly the water mediums properties that causes noise in the depthmap, and that particles are not enlarging this disorder significantly, or if enough particles within reason will disturb the depth measurements done by the Raytrix.

Since we could not get the naturally disturbances in the water to affect the depthmap directly, noise was added to the computed depthmap images.

The noise used is random sized color circles with random color placed randomly. The radius of each pixel is between 1 and 6 pixels. Noise was added from level 1 to 40, where level 1 starts with 20 particles, and each level adds 40 particles. That is, level 40 has 1,580 random particles in the image.

The algorithm has few problems up to level 20. Figure \ref{fig:noise_level_1}, and figure \ref{fig:noise_level_8} shows noise level 1 and 8, respectively, and shows few differences. After level 20, small parts of the fish starts missing (figure \ref{fig:noise_level_22}), and those holes get bigger and more concentrated up to level 40. Still, the algorithm can sometimes get decent results, an example is figure \ref{fig:noise_level_32}. Level 40 is almost covered with particles, and this would most likely never be a real issue, but the result is still not too bad.

If this technology is to be used in conditions containing more particles than from the test case used in this report, it is reasonable to believe that some more processing and tuning could give good results also then. The case would most likely be to fill in the missing parts even more, and perhaps calculate the largest contour from the totalfocus image instead of the depthmap. 





{\color{red} Put this in result???

It is seen from the noisy images that it has almost no effect on the result. This shows that the algorithm is robust and will work in most particle conditions as long as the light settings remains the same and the distance to the fish is within some boundaries. }


% NEW TEST

\begin{figure}[h]
    \centering
    \begin{subfigure}{1\textwidth}
        \centering
        \includegraphics[width=.7\linewidth]{images/implementation/noise/noise63_1}
        \caption{Depthmap image with noise level 1} 
        \label{fig:image_noise_level_1}
    \end{subfigure}\hspace*{\fill}
    
    \medskip
    \begin{subfigure}{1\textwidth}
        \centering
        \includegraphics[width=.7\linewidth]{images/implementation/noise/filternoise63_1}
        \caption{Resulting depthmap} 
        \label{fig:filter_noise_level_1}
    \end{subfigure}\hspace*{\fill}
    \caption{Depthmap image and result for noise level 1}
    \label{fig:noise_level_1}
\end{figure}

\begin{figure}[h]
    \centering
    \begin{subfigure}{1\textwidth}
        \centering
        \includegraphics[width=.7\linewidth]{images/implementation/noise/noise63_8}
        \caption{Depthmap image with noise level 8} 
        \label{fig:image_noise_level_8}
    \end{subfigure}\hspace*{\fill}
    
    \medskip
    \begin{subfigure}{1\textwidth}
        \centering
        \includegraphics[width=.7\linewidth]{images/implementation/noise/filternoise63_8}
        \caption{Resulting depthmap} 
        \label{fig:filter_noise_level_8}
    \end{subfigure}\hspace*{\fill}
    \caption{Depthmap image and result for noise level 8}
    \label{fig:noise_level_8}
\end{figure}

\begin{figure}[h]
    \centering
    \begin{subfigure}{1\textwidth}
        \centering
        \includegraphics[width=.7\linewidth]{images/implementation/noise/noise63_22}
        \caption{Depthmap image with noise level 22} 
        \label{fig:image_noise_level_22}
    \end{subfigure}\hspace*{\fill}
    
    \medskip
    \begin{subfigure}{1\textwidth}
        \centering
        \includegraphics[width=.7\linewidth]{images/implementation/noise/filternoise63_22}
        \caption{Resulting depthmap} 
        \label{fig:filter_noise_level_22}
    \end{subfigure}\hspace*{\fill}
    \caption{Depthmap image and result for noise level 22}
    \label{fig:noise_level_22}
\end{figure}

\begin{figure}[h]
    \centering
    \begin{subfigure}{1\textwidth}
        \centering
        \includegraphics[width=.7\linewidth]{images/implementation/noise/noise63_32}
        \caption{Depthmap image with noise level 32} 
        \label{fig:image_noise_level_32}
    \end{subfigure}\hspace*{\fill}
    
    \medskip
    \begin{subfigure}{1\textwidth}
        \centering
        \includegraphics[width=.7\linewidth]{images/implementation/noise/filternoise63_32}
        \caption{Resulting depthmap} 
        \label{fig:filter_noise_level_32}
    \end{subfigure}\hspace*{\fill}
    \caption{Depthmap image and result for noise level 32}
    \label{fig:noise_level_32}
\end{figure}

\begin{figure}[h]
    \centering
    \begin{subfigure}{1\textwidth}
        \centering
        \includegraphics[width=.7\linewidth]{images/implementation/noise/noise63_40}
        \caption{Depthmap image with noise level 40} 
        \label{fig:image_noise_level_40}
    \end{subfigure}\hspace*{\fill}
    
    \medskip
    \begin{subfigure}{1\textwidth}
        \centering
        \includegraphics[width=.7\linewidth]{images/implementation/noise/filternoise63_40}
        \caption{Resulting depthmap} 
        \label{fig:filter_noise_level_40}
    \end{subfigure}\hspace*{\fill}
    \caption{Depthmap image and result for noise level 40}
    \label{fig:noise_level_40}
\end{figure}

\clearpage

\section{Experiments}\label{experiments}
In this section we describe the results.


\clearpage

\section{Conclusion}\label{conclusion}

The objective of this study was to enhance the depthmap of underwater objects computed by the plenoptic camera Raytrix R42, so it can further be used for a more accurate volume measurement of the object. 

In order to achieve this, an image processing algorithm containing color filtering, morphological operators, object detection and different filtering techniques was introduced. All processing was carried out in RxLive, C++ and OpenCV. MATLAB was used as a visualization tool for better understanding of the results and its improvements. The results showed a great improvement of the depthmap image, with most particles removed and all spots lacking data filled in with appropriate data. MATLAB plots also proves the preservation of depth data.

Different levels of noise were introduced as to illustrate how well the proposed method handles particle noise. The methods implementation was not changed during noise testing, and it showed that the method performed surprisingly well in harsh underwater conditions. 

The results attained from this report were promising as to the further development of volume measuring of underwater objects using the Raytrix plenoptic technology. It is reason to believe that the method developed from this study can be part of a complete future underwater system detecting and measuring biomass of fish in fish farms. 



\clearpage


\bibliographystyle{plain}
\bibliography{kilder}

\end{document}

  